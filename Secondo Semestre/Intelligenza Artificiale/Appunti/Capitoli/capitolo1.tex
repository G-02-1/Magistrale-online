\chapter{Introduzione}
\section{Storia}
L'IA nasce con Turing nel 1950, mentre, il termine "Artificial Intelligence" viene coniato da John Mccarthy. 
\newline
Il Libro consigliato è S.Russel, P.Norvig, Artificial Intelligence: A modern approach, Prentice Hall Ed.
\newline
\textbf{Esame:}
\begin{itemize}
    \item Prova orale.
    \item Progettino.
    \item (Forse) Prova Scritta.
    \item Durante il corso si svolgerà una gara progettuale che da 1 punto bonus partecipazione e 3 punti bonus vittoria.
\end{itemize}
\capitalgrave{E} possibile trovare esercizi e codici su myLab con codice INUWRTQL inserendolo in combinazione al codice del libro acquistato.
\subsubsection{Cos'è l'intelligenza artificiale?}
L'intelligenza artificiale è nelle nostre vite da moltissimo tempo, tuttavia pochi individui ne conoscevano gli sviluppi.
Alcuni esempi possono essere, robot "faccendieri", droni o macchine a guida autonoma.

\newpage
Cos'è la razionalità computazionale? 
\begin{definition}
    \capitalgrave{E} la capacità di raggiungere gli obiettivi prefissati in modo da massimizzarne l'utilità.
\end{definition}
\subsubsection{No-Monotonic-Reasoning}
Che cosa vuol dire "No-Monotonic-Reasoning"? \capitalgrave{E} un termine che si tramanda dalla nascita del pensiero filosofico.
La logica classica è una logica monotona, ovvero, aggiungere nuove conoscenze implica una crescita del numero di conclusioni.
\newline
Tuttavia, una nuova conoscenza otrebbe invalidare le vecchie conoscienze, implicando quella che chiamiamo "eccezione", così i ricercatori
nel campo dell'IA lavorano sul \textbf{Common-Sense Reasoning} raggiungendo lo sviluppo di pensiero non monotòno.
\newline
Quali sono le conseguenze dello sviluppo di un pensiero non monotòno?
\newline
La conseguenza principale è quella dell'aumento della complessità computazionale degli algoritmi decisionali, arrivando anche a raggiungere 
classi di co mplessità agli apici della piramide dell'hardness.
Il vantaggio stà nel fatto che, questi linguaggi, sono estremamente espressivi.  
\newline
Che intendiamo con "espressivo"?
\begin{definition}
    In questo caso, definiziamo l'espressività come la capacità di un linguaggio di riuscire ad esprimere tutti i problemi della 
    stessa classe di complessità alla quale appartiene la computazione, legata alla risoluzione dei problemi intrinseci nel 
    linguaggio in analisi.
\end{definition}
Intercorriamo in atri due concetti di cruciale importanza:
\begin{itemize}
    \item Complessità. \textit{Quanto è potente questa logica? Che complessità di problemi possiamo affrontare con questa logica?}
    \item Espressività. \textit{Quanti problemi posso esprimere con questa logica?}
\end{itemize}
La teoria dell'aspressività ci consente di comprendere quali sono i problemi risolvibili con la logica (nello specifico contesto corrente) in analisi.

\newpage
Lo studio dell'Intelligenza Artificiale, fra l'altro, coinvolge lo studio della risoluzione di problemi complessi a livello prettamente algoritmico.
Durante il corso, verranno esplorate vari temi cruciali come: 
\begin{itemize}
    \item Complessità strutturale.
    \item Ipergrafi.
    \item Teoria dei giochi (esempio della cooperazione dei detenuti, concetto strettamente connesso al "problema di Nash").
    \item Fair Division And Fair Allocation Problem
\end{itemize}
